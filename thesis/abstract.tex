\svnInfo $Id: main.tex 54 2008-03-31 18:26:19Z axel $
\chapter*{\centering Zusammenfassung}

Raytracing ist ein grundlegengendes Verfahren der Computergrafik zur Synthese fotorealistischer Bilder. Die Simulation einzelner Lichtpfade ermöglicht es aus einer abstrakten Beschreibung einer dreidimensionalen Szene ein zweidimensionales Bild zur Darstellung am Computermonitor zu generieren. Die Verfolgung der Strahlen die den Lichtfluss simulieren ist ein sehr rechenaufwändiger Vorgang. Bis vor wenigen Jahren wurde deswegen nicht in Erwägung gezogen Raytracing für die Bildgenerierung in Echtzeit einzusetzen. Die steigende Rechenleistung der Prozessoren und die Entwicklung effizienter Algorithmen ermöglichen mittlierweile die Visualisierung virtueller Szenen mit dem Raytracingverfahren auf aktuellen Arbeitsplatzrechnern. Globale Beleuchtung und dynamische Szenen eröffnen eine neue Problemklasse die noch effizientere Ausnutzung von Kohärenzen der Szenendaten und der aktuellen Hardware fordert. Die langjähre Forschungsarbeit auf diesem Gebiet hat eine enorme Menge Literatur zu diesem Thema hervorgebracht. Diese Arbeit gibt eine Einführung in die wichtigsten Techniken die für die Implementierung eines Echtzeit Raytracers benötigt werden. Außerdem sollen die Grundlagen für das Verständnis aktueller Publikationen gelegt werden, welche die Kenntnis dieser Techniken voraussetzen.
Im Zusammenhang dieser Arbeit wurde ein offenes Framework erstellt, in dem viele der vorgestellten Verfahren leicht verständlich implementiert wurden. Durch hochoptimierten Code verbergen existierende, frei verfügbare Implementierungen oft die Arbeitsweise der eigentlichen Algorithmen. Bei dem mit dieser Arbeit erstellten Quellcode wurde besonderer Wert auf eine klare Struktur und gute Nachvollziehbarkeit der implementierten Algorithmen gelegt, ohne dabei auf den Anspruch interaktiver Frameraten zu verzichten.
