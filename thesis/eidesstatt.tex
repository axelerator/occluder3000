% Die eidesstattliche Erklärung auf einer neuen Seite
\newpage
% Keine Nummerierung f�r diesen Teil
\chapter*{Eidesstattliche Erklärung}
% Ins Inhaltsverzeichnis aufnehmen
\addcontentsline{toc}{chapter}{Eidesstattliche Erklärung}
% Keine Kopf- und Fußzeilen
\thispagestyle{empty}
% Hier der offizielle Text der eidesstattlichen Erklärung
Ich erkläre hiermit an Eides Statt, dass ich die vorliegende Arbeit selbstständig und ohne Benutzung anderer als der angegebenen Hilfsmittel angefertigt habe; die aus fremden Quellen direkt oder indirekt übernommenen Gedanken sind  als solche kenntlich gemacht. Die Arbeit wurde bisher in gleicher oder ähnlicher Form keiner anderen Prüfungskommission vorgelegt und auch nicht veröffentlicht.
% Etwas Abstand f�r die Unterschrift
\vspace{3cm}
% Hier kommt die Unterschrift dr�ber
\begin{tabbing}
\hspace{6cm}  \= \kill
\textit{Ort, Datum} \> \textit{Axel Tetzlaff}
\end{tabbing}